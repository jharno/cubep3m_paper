\section{Poisson Solver}
\label{sec:Poisson}


This section reviews how Poisson's equation is solved on a double-mesh configuration. 
Many parts of the algorithm are identical to {\small PMFAST}, hence we refer the reader 
to section 2 of MPT for more details. In {\small CUBEP3M}, the mass default assignment scheme are
a `cloud-in-cell' (CIC) interpolation for the coarse grid,  and a `nearest-grid-point' (NGP) interpolation 
for the fine grid \citep{1981csup.book.....H}. This choice is motivated by the fact that the most straightforward 
way to implement a p$^3$m algorithm on a mesh is to have exactly zero mesh force inside a grid, 
which is only true for the NGP interpolation. Although CIC generally has a smoother and more accurate force,
the pp implementation enhances the code resolution almost by an order of magnitude. 
We impose a softening length of one tenth of a grid cell to prevent
large scattering; particles seperated by less than this distance have their 
particle-particle force set to zero.

The code units inherit from \citep{2004NewA....9..443T}, which we summarize here for completeness.
In simulation units,  the comoving length of a fine grid cell is set to one,
such that the unit length in simulation unit is 
\begin{eqnarray}
1\mathcal{L} = a \frac{L}{N} 
\end{eqnarray}
where $a$ is the scale factor, $N$ is the total number of cells along one dimension,
and $L$ is the comoving volume in $h^{-1}\mbox{Mpc}$.
The mean comoving mass density is also set to unity in simulation units, 
which, in physical units, corresponds to 
\begin{eqnarray}
1\mathcal{D} = \rho_{m}(0) a^{-3} = \Omega_{m} \rho_{c} a^{-3} = \frac{3 \Omega_{m} H_{o}^{2}}{8 \pi G a^{3} }
\end{eqnarray}
$\Omega_{m}$ is the matter density, $H_{o}$ is the Hubble's constant, $\rho_{c}$ is the critical density today,
and $G$ is Newton's constant. The mass unit is found with $\mathcal{M} = \mathcal{DL}^{3}$.
Specifying the value of $G$ on the grid fixes the time unit, and with the choice $G_{grid}$ = 1/$(6 \pi a)$,
we get:
\begin{eqnarray}
1 \mathcal{T} = \frac{2a^{2}}{3}\frac{1}{\sqrt{\Omega_{m}H_{o}^{2}}}
\end{eqnarray}
These choices completely determine the convertion between physical units into simulation units and vice versa.
For instance, the velocity units are given by $1\mathcal{V} = \mathcal{L}$/$\mathcal{T}$.



The force of gravity on a mesh can be computed either with a gravitational potential  kernel $\omega_{\phi} ({\bf x})$
  or a force  kernel $\omega_{F} ({\bf x})$.
Gravity fields are curl-free, which allows us to relate the potential $\phi({\bf x})$ to the source term via Poisson's equation: 
\begin{eqnarray}
\nabla^{2}\phi({\bf x}) = 4 \pi G \rho({\bf x})
\label{eq:poisson}
\end{eqnarray}
$G$ being Newton's constant. We solve this equation in Fourier space, where we write
\begin{eqnarray}
 \tilde{\phi}({\bf k}) = \frac{4 \pi G \tilde{\rho}({\bf k})}{- k^{2}} \equiv \tilde{\omega}_{\phi}({\bf k})\tilde{\rho}({\bf k})
\label{eq:poissonFourier}
\end{eqnarray}
The potential in real space is then obtained with an inverse Fourier transform, and the kernel becomes $\omega_{\phi} ({\bf x}) = -G/r$.
Using the convolution theorem, we can write
\begin{eqnarray}
 \phi({\bf x}) = \int \rho({\bf x'}) \omega_{\phi}({\bf x'} - {\bf x}) d{\bf x'}
\label{eq:poisson_solution_pot}
\end{eqnarray}
Although this approach is fast, it involves a finite differentiation which enhances the numerical noise.
We therefore opt for a force kernel, which is more accurate but has the inconvenient to require four extra Fourier transforms.
In this case, we must solve the convolution in three dimensions:
\begin{eqnarray}
 F({\bf x}) = - m {\bf \nabla} \phi({\bf x})   = \int \rho({\bf x'}) \mbox{\boldmath $\omega$}_{F}({\bf x'} - {\bf x}) d{\bf x'}                                      
\label{eq:poisson_solution_force}
\end{eqnarray}
The differentiation does not affect the density since it only acts on un-prime variables,
and the force kernel is given by 
\begin{eqnarray}
\mbox{\boldmath $\omega$}_{F}({\bf x}) \equiv - {\bf \nabla}\omega_{\phi}({\bf x}) = - \frac{mG \hat{\bf r}}{r^{2}}
\end{eqnarray}

Following the spherically symmetric matching technique of MPT (section 2.1), 
we split  the force kernel into two components, for the short and long range respectively, and 
match the overlapping region with a polynomial. Namely, we have:
\begin{eqnarray}
\mbox{\boldmath $\omega$}_{s}(r) = \begin{cases} \mbox{\boldmath $\omega$}_{F}(r) -  \mbox{\boldmath $\beta$}(r) &\mbox{if  } \mbox{$r$ $\le$ $r_{c}$ } \\
0 & \mbox{otherwise} 
\end{cases}
\end{eqnarray}
and
\begin{eqnarray}
\mbox{\boldmath $\omega$}_{l}(r) = \begin{cases} \mbox{\boldmath $\beta$}(r) &\mbox{if  } \mbox{$r$ $\le$ $r_{c}$ } \\
 \mbox{\boldmath $\omega$}_{F}(r)  &\mbox{otherwise} 
\end{cases}
\end{eqnarray}
The vector $\mbox{\boldmath $\beta$}(r)$ is related to the fourth order polynomial that is used in the potential case by
 $ \mbox{\boldmath $\beta$} = - {\bf \nabla} \alpha(r)$. The coefficients are found by matching the boundary conditions at $r_{c}$ up to the second derivative,
 and we get
  \begin{eqnarray}
   \mbox{\boldmath $\beta$}(r) = \bigg[ -\frac{7 r}{4 r_{c}^{3}} + \frac{3 r^{3}}{4 r^{5}}\bigg] \hat{\bf r}
  \end{eqnarray}

Because these calculations are performed on two grids of different resolution, a sampling window function must be convoluted 
both with the density and the kernel (see [Eq. 7-8] of MPT).
When matching the two  force kernels, it was realized that close to the cutoff region, the long range force is always on the low side, whereas 
the short range force is scattered across the theoretical $1/r^2$ value. These behaviors are purely features of the CIC and NGP interpolation scheme 
respectively. We identified a small range surrounding the cutoff length, in which we empirically adjust both kernels such as to improve 
the match. Namely, for $14 \le r \le 16$, $\mbox{\boldmath $\omega$}_{s}({r}) \rightarrow \mbox{\boldmath $\omega$}_{s}({ r})*0.985$,
and for  $12 \le r \le 16$, $\mbox{\boldmath $\omega$}_{l}({r}) \rightarrow \mbox{\boldmath $\omega$}_{l}({ r})*1.2$.
The two fudge factors were found by performing force measurements on two particles randomly placed in the volume.

In addition, {\small PMFAST} could run with a different kernel, described in MPT as the `least square kernel', 
which basically adjust the kernel on cell by cell basis based on minimization of the deviation with respect
to the Newtonian predictions. This was originally computed such as to optimize the force calculation for
the case where both grids are computed from CIC interpolation. Moving to a mix CIC/NGP scheme
requires solving the system of equations with the new configuration, a straightforward operation. 
{\bf (results are presented somewhere?)}

Finally, a choice must be done concerning the longest range of the force. Gravity can be either a) an accurate $1/r^2$ force, as far as the volume allows, 
or b) modified to correctly match the periodicity of the boundary conditions. By default, the code is configured along to the second choice,
which accurately models the growth of structure at very large scales. However, detailed studies of gravitational collapse would benefit 
from the first settings.

{\bf Least square kernel?}
