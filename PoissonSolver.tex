\section{Poisson Solver}
\label{sec:Poisson}


This section describes how Poisson's equation is solved at each time step of the main loop. 
Many parts of the algorithm are identical to {\small PMFAST}, hence we refer extensively 
to section 2 of MPT.

The reason that motivates the double mesh configuration is to allow the force of gravity to be broken down into 
a long and a short range components. This two-mesh decomposition was first seen in Hydra \citep{hydra}, 
and is now used by many codes, including {\small PMFAST}, {\bf (TracPen2003)} and {\bf (Other codes?)}.
The long range force requires knowledge of the complete simulated volume, 
but can be computed on a coarser grid without degrading significantly the precision of the calculation.
Using a grid 64 times coarser limits the global communication time, plus contribute only to a fraction of the total memory requirement.
The short range force is computed on the finer mesh and is local in the sense that it only knows about the matter in the immediate surroundings.
It can thus be computed simultaneously on many computing units with {\small OPENMP} parallel loops, each {\small CPU} solving for a different sub-volume,
as long as the arrays are constructed such as to support parallel reading and writing\footnote{In practice, this is done by adding an additional dimension
to the relevant arrays, such that each {\small CPU} accesses a unique memory location.}. 
The particle-particle (pp) interactions are also local operations, and are computed inside the same {\small OPENMP} loops. 

The force of gravity on a mesh can be computed either with a gravitational potential  kernel $\omega_{\phi} ({\bf x})$
  or a force  kernel $\omega_{F} ({\bf x})$.
It is a curl-free field, which allows us to relate the potential $\phi({\bf x})$to the source term via Poisson's equation: 
\begin{eqnarray}
\nabla^{2}\phi({\bf x}) = 4 \pi G \rho({\bf x})
\label{eq:poisson}
\end{eqnarray}
$G$ being Newton's constant. We solve this equation in Fourier space, where we write
\begin{eqnarray}
 \tilde{\phi}({\bf k}) = \frac{4 \pi G \tilde{\rho}({\bf k})}{- k^{2}} \equiv \tilde{\omega}_{\phi}({\bf k})\tilde{\rho}({\bf k})
\label{eq:poissonFourier}
\end{eqnarray}
The potential in real space is then obtained with an inverse Fourier transform, and the kernel becomes $\omega_{\phi} ({\bf x}) = -G/r$.
Using the convolution theorem, we can write
\begin{eqnarray}
 \phi({\bf x}) = \int \rho({\bf x'}) \omega_{\phi}({\bf x'} - {\bf x}) d{\bf x'}
\label{eq:poisson_solution_pot}
\end{eqnarray}
Although this approach is fast, it involves a finite differentiation which enhances the numerical noise.
We therefore opt for a force kernel, which is more accurate but has the inconvenient to require four extra Fourier transforms.
In this case, we must solve the convolution in three dimensions:
\begin{eqnarray}
 F({\bf x}) = - m {\bf \nabla} \phi({\bf x})   = \int \rho({\bf x'}) {\bf  \omega}_{F}({\bf x'} - {\bf x}) d{\bf x'}                                      
\label{eq:poisson_solution_force}
\end{eqnarray}
The differentiation does not affect the density since it only acts on un-prime variables,
and the force kernel is given by 
\begin{eqnarray}
{\bf  \omega}_{F}({\bf x}) \equiv - {\bf \nabla}\omega_{\phi}({\bf x}) = - \frac{mG \hat{\bf r}}{r^{2}}
\end{eqnarray}

Following MPT, we split  the force kernel into two components, for the short and long range respectively, and 
match the overlapping regions with a polynomial. Namely, we have:
\begin{eqnarray}
{\bf \omega}_{s}(r) = \begin{cases} {\bf \omega}_{F}(r) -  {\bf \beta}(r) &\mbox{if  } \mbox{$r$ $\le$ $r_{c}$ } \\
0 & \mbox{otherwise} 
\end{cases}
\end{eqnarray}
and
\begin{eqnarray}
{\bf \omega}_{l}(r) = \begin{cases} {\bf \beta}(r) &\mbox{if  } \mbox{$r$ $\le$ $r_{c}$ } \\
 {\bf\omega}_{F}(r)  &\mbox{otherwise} 
\end{cases}
\end{eqnarray}
The vector $ {\bf \beta}(r)$ is related to the fourth order polynomial that is used in the potential case by
 $ {\bf \beta} = - {\bf \nabla} \alpha(r)$. The coefficients are found by matching the boundary conditions at $r_{c}$ up to the second derivative,
 and we get
  \begin{eqnarray}
   {\bf \beta}(r) = \bigg[ -\frac{7 r}{4 r_{c}^{3}} + \frac{3 r^{3}}{4 r^{5}}\bigg] \hat{\bf r}
  \end{eqnarray}
The long range force is subject to a correction that fixes {\bf (what exactly?)}.

Window, kernel matching...

 equation to match them, fudge factor,
corrected kernel, pp and extended pp.
