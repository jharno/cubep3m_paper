\subsection{Particle identification tags}
\label{PID}

A system of particle identification can be turned on, which basically allows to track each particle's trajectory
between checkpoints. Such a tool is useful for a number of applications, from reconstruction of halo merging history to tracking individual particle
trajectories.
The particle tag system has been implemented as an array of double integers, {\tt PID}, 
and assigns a unique integer to each particle during the initialization stage. The location of the tag on the {\tt PID} array 
matches the location of the corresponding particle on the {\tt xv} array, hence it acts as if the latter array had an extra dimension.
The location change only when particles exist the local volume, in which case the tag is sent along adjacent nodes 
with the particle in the {\tt pass\_particle} subroutine, and deleted particles result in deleted flags.
Similarly to the phase space array, the {\tt PID} array gets written to file at each particle checkpoint.
