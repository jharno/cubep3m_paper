\section{Runtime Halo Finder}
\label{sec:halo}

%Spherical overdensities, search algorithm, 
%provided halo information, halo bias, comparison to PS and ST.
%{\bf (Ilian, You can lead the way here...)}

We have implemented a halo finding procedure, which we have developed 
based on the spherical overdensity (SO) approach \citep{1994MNRAS.271..676L}.
In the interest of speed and efficiency the halo catalogues are constructed 
on-the-fly at a pre-determined list of redshifts. The halo finding is 
massively-parallel and threaded based on the CubeP$^3$M data structures 
discussed earlier. The code first builds the 
fine-mesh density for each sub-domain using CIC or NGP interpolation. It then 
proceeds to search for and record all local density maxima above certain
threshold (typically set to 100 above mean density) within the local 
sub-domain's physical volume (excluding the sub-domain buffer zones). It then 
uses parabolic interpolation on the density field to determine more precisely
the location of the maximum within the densest cell, and records the peak 
position and value. The halo center determined this way agrees closely with 
the center-of-mass of the halo particles discussed below.  

Once the list of peak positions is generated, they are sorted from the highest 
to the lowest density value. Then each of the halo candidates is inspected 
independently, starting with the highest peak. The grid mass is accumulated 
in spherical shells of fine grid cells surrounding the maximum, until the 
mean density within the halo drops below a pre-defined overdensity cutoff 
(usually set to 178 in units of the mean, in accordance to the top-hat 
collapse model). As we accumulate mass we remove it from the mesh, so that no 
mass element is double-counted. This method is thus inappropriate for finding 
sub-halos as within this framework those are naturally incorporated in their 
host halos. Because the halos are found on a grid of finite-size cells, it is 
possible, especially for the low-mass halos, to overshoot the target overdensity.
When this occurs we use an analytical halo density profile to correct the 
halo mass and radius to the values corresponding to the target overdensity. 
This analytical density profile is given by Truncated Isothermal Sphere (TIS) 
profile \citep{1999MNRAS.307..203S,2001MNRAS.325..468I} for overdensities below 
$\sim130$, and $1/r^2$ for lower overdensities. The TIS density profile has a
similar outer slope (the relevant one here) to the Navarro, Frenk and White 
\citep[NFW][]{1997ApJ...490..493N} profile, but extends to lower overdensities
and matches well the virialization shock position given by the Bertschinger 
self-similar collapse solution \citep{1985ApJS...58...39B}.

Once the correct halo mass, radius and position are determined, we find all 
particles which are within the halo radius. Their positions and velocities are
used to calculate the halo center-of-mass, bulk velocity, internal velocity 
dispersion and the three angular momentum components, all of which are then 
included in the final halo catalogues. We also calculate the total mass of
all particles within the halo radius, also listed in the halo data. This mass
is very close, but  typically slightly lower, than the halo mass calculated 
based on the gridded density field. The particle center-of-mass corresponds 
very well to the halo centre found based on the gridded mass distribution. 

{\bf II: do we need to say anything about having the option to use other
halo finders like AHF and FOF?}

{\bf Show some examples and comparisions to analytical fits and other halo 
finders.}
