\section{Conclusion}

This paper describes {\small CUBEP3M}, a public P$^3$M N-body code that is massively parallel, memory local, 
and that scales well to 20,000 cores, thus raising the limits of the cosmological problem size one can handle.
We summarize the code structure, the two-mesh Poisson solver algorithm, we present scaling and systematic tests
that have been performed. We also describe a few utilities and extensions that come with the public release, 
including a run time halo finder, an extended pp force calculation and a non-Gaussian initial condition generator.

The code is publicly available on github.com under {\tt cubep3m}, and extra documentation about the structure, 
compiling and running strategy is can be found at {\tt www.wiki.cita.utoronto.ca/mediawiki/index.php/CubePM}.

\section*{Acknowledgements}

The CITA256 simulations were run on the Sunnyvale cluster at CITA.
ITI was supported by The Southeast Physics
Network (SEPNet) and the Science and Technology Facilities Council
grants ST/F002858/1 and ST/I000976/1. The authors acknowledge the
TeraGrid and the Texas Advanced Computing Center (TACC) at The
University of Texas at Austin (URL: http://www.tacc.utexas.edu) for
providing HPC and visualization resources that have contributed to the
research results reported within this paper. ITI also acknowledges the Partnership for Advanced
Computing in Europe (PRACE) grant 2010PA0442 which supported the code
scaling studies. 