\section{Conclusion}

This paper describes {\small CUBEP3M}, a public and  massively parallel P$^3$M N-body code that inherits from {\small PMFAST} 
 and that now scales well to 20,000 cores, pushing the limits of the cosmological problem size one can handle.
We summarize the code structure, review the double-mesh Poisson solver algorithm, and present scaling and systematic tests
that have been performed. We also describe various utilities and extensions that come with the public release, 
including a run time halo finder, an extended pp force calculation and a non-Gaussian initial condition generator.
{\small CUBEP3M} is one of the most competitive N-body code that is publicly available for cosmologists and astrophysicists,
it has already been used  for a large number of scientific applications, and it is our hope that the current documentation will 
help the community in interpreting its outcome.
The code is publicly available on github.com under {\tt cubep3m}, and extra documentation about the structure, 
compiling and running strategy is can be found on the CITA wiki page\footnote{\tt www.wiki.cita.utoronto.ca/mediawiki/index.php/CubePM}.

\section*{Acknowledgements}

The CITA simulations were run on the Sunnyvale cluster at CITA.
ITI was supported by The Southeast Physics
Network (SEPNet) and the Science and Technology Facilities Council
grants ST/F002858/1 and ST/I000976/1. 
Computations for the SciNet1024 runs were performed on the GPC supercomputer at the SciNet HPC Consortium. SciNet is funded by: the Canada Foundation for Innovation under the auspices of Compute Canada; the Government of Ontario; Ontario Research Fund - Research Excellence; and the University of Toronto. 
The authors acknowledge the
TeraGrid and the Texas Advanced Computing Center (TACC) at The
University of Texas at Austin (URL: http://www.tacc.utexas.edu) for
providing HPC and visualization resources that have contributed to the
research results reported within this paper. ITI also acknowledges the Partnership for Advanced
Computing in Europe (PRACE) grant 2010PA0442 which supported the code
scaling studies. UEP and JDE are supported by the NSERC of Canada.
%{\bf ( Hugh, Vincent, JD, any financial acknowledgements to add here?)}

