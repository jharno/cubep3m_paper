
The original code that provides Gaussian initial conditions for {\small CUBEP3M} 
is extended to include non-Gaussian features of the ``local'' form,
$\Phi({\bf x})=\phi({\bf x})+f_{\rm NL} \phi({\bf x})^2 + g_{\rm NL} 
\phi({\bf x})^3$, where $\phi({\bf x})$ is the Gaussian contribution
to the Bardeen potential $\Phi({\bf x})$ (see \cite{F_NL} for a review). 
We adopted the CMB convention,
in which $\Phi$ is calculated immediately after the matter-
radiation equality (and not at redshift $z=0$ as in the large scale
structure convention). For consistency, $\phi({\bf x})$ is normalized
to the amplitude of scalar perturbations inferred by CMB measurements
($A_s\approx 2.2 \times 10^{-9}$). The local transformation is performed 
before the inclusion of the matter transfer function, and the initial 
particle positions and velocities are finally computed from $\Phi({\bf x})$ 
according to the Zel'dovich approximation, as in the original Gaussian initial condition generator.

This code was tested by comparing simulations and theoretical predictions
for the effect of local primordial non-Gaussianity on the halo mass 
function and matter power spectrum (Desjacques, Seljak \& Iliev 2009). 
It has also been used to quantify the impact of local non-Gaussian initial
conditions on the halo power spectrum \citep{2009MNRAS.396...85D,
2010PhRvD..81b3006D} and bispectrum \citep{2010MNRAS.406.1014S},
 as well as the matter bispectrum \citep{2011arXiv1111.6966S}.

